% Options for packages loaded elsewhere
\PassOptionsToPackage{unicode}{hyperref}
\PassOptionsToPackage{hyphens}{url}
%
\documentclass[
]{book}
\usepackage{amsmath,amssymb}
\usepackage{lmodern}
\usepackage{ifxetex,ifluatex}
\ifnum 0\ifxetex 1\fi\ifluatex 1\fi=0 % if pdftex
  \usepackage[T1]{fontenc}
  \usepackage[utf8]{inputenc}
  \usepackage{textcomp} % provide euro and other symbols
\else % if luatex or xetex
  \usepackage{unicode-math}
  \defaultfontfeatures{Scale=MatchLowercase}
  \defaultfontfeatures[\rmfamily]{Ligatures=TeX,Scale=1}
\fi
% Use upquote if available, for straight quotes in verbatim environments
\IfFileExists{upquote.sty}{\usepackage{upquote}}{}
\IfFileExists{microtype.sty}{% use microtype if available
  \usepackage[]{microtype}
  \UseMicrotypeSet[protrusion]{basicmath} % disable protrusion for tt fonts
}{}
\makeatletter
\@ifundefined{KOMAClassName}{% if non-KOMA class
  \IfFileExists{parskip.sty}{%
    \usepackage{parskip}
  }{% else
    \setlength{\parindent}{0pt}
    \setlength{\parskip}{6pt plus 2pt minus 1pt}}
}{% if KOMA class
  \KOMAoptions{parskip=half}}
\makeatother
\usepackage{xcolor}
\IfFileExists{xurl.sty}{\usepackage{xurl}}{} % add URL line breaks if available
\IfFileExists{bookmark.sty}{\usepackage{bookmark}}{\usepackage{hyperref}}
\hypersetup{
  pdftitle={NYU Child Language Lab Manual},
  hidelinks,
  pdfcreator={LaTeX via pandoc}}
\urlstyle{same} % disable monospaced font for URLs
\usepackage{longtable,booktabs,array}
\usepackage{calc} % for calculating minipage widths
% Correct order of tables after \paragraph or \subparagraph
\usepackage{etoolbox}
\makeatletter
\patchcmd\longtable{\par}{\if@noskipsec\mbox{}\fi\par}{}{}
\makeatother
% Allow footnotes in longtable head/foot
\IfFileExists{footnotehyper.sty}{\usepackage{footnotehyper}}{\usepackage{footnote}}
\makesavenoteenv{longtable}
\usepackage{graphicx}
\makeatletter
\def\maxwidth{\ifdim\Gin@nat@width>\linewidth\linewidth\else\Gin@nat@width\fi}
\def\maxheight{\ifdim\Gin@nat@height>\textheight\textheight\else\Gin@nat@height\fi}
\makeatother
% Scale images if necessary, so that they will not overflow the page
% margins by default, and it is still possible to overwrite the defaults
% using explicit options in \includegraphics[width, height, ...]{}
\setkeys{Gin}{width=\maxwidth,height=\maxheight,keepaspectratio}
% Set default figure placement to htbp
\makeatletter
\def\fps@figure{htbp}
\makeatother
\setlength{\emergencystretch}{3em} % prevent overfull lines
\providecommand{\tightlist}{%
  \setlength{\itemsep}{0pt}\setlength{\parskip}{0pt}}
\setcounter{secnumdepth}{5}
\ifluatex
  \usepackage{selnolig}  % disable illegal ligatures
\fi

\title{NYU Child Language Lab Manual}
\author{}
\date{\vspace{-2.5em}}

\begin{document}
\maketitle

{
\setcounter{tocdepth}{1}
\tableofcontents
}
\hypertarget{welcome}{%
\chapter{Welcome}\label{welcome}}

Welcome to NYU's Child Language Lab Wiki! Click on tabs to the left to explore our site!

\hypertarget{links}{%
\section*{Links}\label{links}}
\addcontentsline{toc}{section}{Links}

\begin{itemize}
\item
  \href{https://wp.nyu.edu/childlanguagelab/}{Lab website}
\item
  \href{https://wp.nyu.edu/cournane/}{PI Website}
\item
  \href{https://www.facebook.com/NYUcll/}{Lab Facebook Page}
\item
  Location: Room 208 and 209 (Floor 2), 10 Washington Place, Department of Linguistics
\end{itemize}

\hypertarget{mission-statement}{%
\section*{Mission Statement}\label{mission-statement}}
\addcontentsline{toc}{section}{Mission Statement}

We are interested in better understanding how young children workout the complex language learning problems they are faced with, specifically abstract words that cannot be taught or pointed to (e.g., how do you show a child the meaning of \emph{must} or \emph{think}?). To do so, we combine expertise in linguistics, cognitive science and developmental psychology to help us understand just how children accomplish this feat. Although neither you nor your child will personally receive any direct benefits from this research, by learning more about children's abstract language development,this research contributes directly to our knowledge about human linguistic and cognitive development, and indirectly to the improvement of early language assessment and intervention.

\hypertarget{current-projects}{%
\section[Current Projects ]{\texorpdfstring{Current Projects \footnote{For more information, see \protect\hyperlink{expov}{this detailed list of experiments}}}{Current Projects }}\label{current-projects}}

\hypertarget{pi-led-projects}{%
\subsection*{PI-led Projects}\label{pi-led-projects}}
\addcontentsline{toc}{subsection}{PI-led Projects}

\begin{enumerate}
\def\labelenumi{\arabic{enumi}.}
\tightlist
\item
  \textbf{Fuzzy NomNoms}: Investigates the acquisition and comprehension of the modal adverb \emph{maybe} in 2-3 year old children.
\item
  \textbf{AllAboutMe}: Investigates the evidential component of \emph{must} and \emph{might} in 5 - 9 year olds \& adults.
\item
  \textbf{ModForceJunior}: Aims to understand how 3.5 - 4.5 year olds, and adults, navigate modal force in acquisition.
\item
  \textbf{SqueakyMice}: Investigates the psycho-linguistic factors influencing 5-7 year olds \& adult reasoning with polysemous modal verbs.
\end{enumerate}

\hypertarget{graduate-student-led-projects}{%
\subsection*{Graduate Student-led Projects}\label{graduate-student-led-projects}}
\addcontentsline{toc}{subsection}{Graduate Student-led Projects}

\begin{enumerate}
\def\labelenumi{\arabic{enumi}.}
\tightlist
\item
  \textbf{CountKiddo} (Maxime): Aims to understand the acquisition path of counterfactual assertions in 4-6 year olds.
\item
  \textbf{Exhaustives} (Alicia): Aims to pindown the source of number inferences in 4-6 year olds.
\item
  \textbf{KidBiComp} (Sarah): Aims to understand the nature and processing of code-switched utterances in 2-year-old English-Spanish bilinguals.
\item
  \textbf{CFWaterpark} (Ioana):
\end{enumerate}

\hypertarget{lab-members}{%
\section{Lab Members}\label{lab-members}}

\hypertarget{pi}{%
\subsection*{PI}\label{pi}}
\addcontentsline{toc}{subsection}{PI}

Dr.~Ailis Cournane

\hypertarget{lab-manager}{%
\subsection*{Lab manager}\label{lab-manager}}
\addcontentsline{toc}{subsection}{Lab manager}

(past): Adam Bell
(current): Charlotte McFarland

\hypertarget{graduate-students}{%
\subsection*{Graduate Students}\label{graduate-students}}
\addcontentsline{toc}{subsection}{Graduate Students}

Maxime Tulling; Alicia Parrish; Sarah Phillips; Chiara Repetti-Ludlow; Anna Alsop; Paloma Jeretič; Naomi Lee; Mary Robinson

\hypertarget{ras}{%
\subsection*{RAs}\label{ras}}
\addcontentsline{toc}{subsection}{RAs}

Hannah Mattis-Roesch; Madisen Fong; Mark Bacon; Maya Wallis; Maya Orey; Isha Rahman;

\hypertarget{interns}{%
\subsection*{Interns}\label{interns}}
\addcontentsline{toc}{subsection}{Interns}

\hypertarget{alumni}{%
\subsection*{Alumni}\label{alumni}}
\addcontentsline{toc}{subsection}{Alumni}

Vishal Sunil Arvindam; Dunja Veselinovic; Hayden Kee; Yohei Oseki; Deborah Hapern; Kathyrn Rafailov; Jenna Pollan; Sasha Frangulov; Daniella Presti; Sam Mitchell; Rachel Arbacher; Michael Marinaccio; Melissa Rojas; Max Manicone; Stacy Gerchick; Nicolette Cure; Michelle Golden

\hypertarget{lab-expectations-and-behavior}{%
\chapter{Lab Expectations and Behavior}\label{lab-expectations-and-behavior}}

The Child Language Lab @ NYU strives to make it a safe, enjoyable, and friendly experience for everyone who participates. We are committed to providing a welcoming and harassment-free experience for everyone, regardless of gender, gender identity and expression, age, sexual orientation, disability, physical appearance, body size, race, ethnicity, religion. We do not tolerate harassment of members in any form.

\hypertarget{general-expectations}{%
\section{General Expectations}\label{general-expectations}}

\begin{enumerate}
\def\labelenumi{\arabic{enumi}.}
\item
  Always follow instructions given to you by the Principal Investigator (PI), the Lab Manager, or the appointed project leader for sub-projects. If you are unsure of something, please ask for clarification. Our aim is to foster clear and open communication for CLL projects.
\item
  Clean-up after yourself. Always try to leave the lab more clean and more organized (or at least as clean and as organized) than when you arrived. Please remember that all spaces are shared, so don't leave things out -- put personal items you would like to keep in the lab in the drawer spaces in Room 208. You are welcome to use the 2nd floor lounge kitchen to store food for the workday, make coffee, etc. If you do, label all food and be considerate of the shared nature of the kitchen space.
\item
  Be responsible \& professional. For example, If you make a mistake on a task, don't hide it as that will make the situation worse. Let the PI (or project lead, or both) know as soon as possible.
\item
  The lab is a team, we work together to get research projects designed, piloted, executed, presented and published. A lab member may be asked to take on some duties that are good for the lab as a whole without direct advantage to themselves. Similarly, one may be asked to put off a task or switch task priorities for the good of the lab. Be a team player.
\item
  If you are having a problem with someone in the lab, please talk to them about it rather than talking to anyone else about it. Only if you are dissatisfied with the outcome, talk to the lab PI about it.
\item
  Treat all lab equipment with respect and care. Be courteous and ensure to charge the iPads, Apple Pencil, and Bluetooth keyboards and mice as needed.
\end{enumerate}

\hypertarget{lab-hierarchy}{%
\section{Lab Hierarchy}\label{lab-hierarchy}}

\begin{enumerate}
\def\labelenumi{\arabic{enumi}.}
\tightlist
\item
  Principle Investigator (PI)
\item
  Lab Manager, PhD Student
\item
  Project Leader
\item
  Undergraduate RA, Intern
\end{enumerate}

\hypertarget{when-and-for-how-long-people-work}{%
\subsection{When and for how long people work}\label{when-and-for-how-long-people-work}}

\begin{itemize}
\tightlist
\item
  RA (Paid): \textasciitilde15 hrs/week
\item
  RA (Research; unpaid): \textasciitilde12 hrs/week
\item
  Intern (Unpaid) \textasciitilde8 hrs/week
\item
  Lab Manager (Part-time): \textasciitilde6 hrs/week
\item
  Lab Manager (Full-time): 35 hrs/week
\end{itemize}

\hypertarget{researchgraduate-student-expectations}{%
\section[Research/Graduate Student Expectations ]{\texorpdfstring{Research/Graduate Student Expectations \footnote{For a comprehensive list of Graduate Student Expectations, refer to the document entitled CLL\_GraduateAdvisorExpectations in the lab Dropbox}}{Research/Graduate Student Expectations }}\label{researchgraduate-student-expectations}}

\begin{enumerate}
\def\labelenumi{\arabic{enumi}.}
\item
  It is important that Researchers stay up to date on the primary literature in their area of study. While the PI will point you to relevant readings, each researcher should make an effort to read articles in scientific journals that specifically apply to their interests and projects.
\item
  Researchers are expected to keep Project Managers and the PI updated on their progress, successes, and concerns. One-on-one meetings can be arranged with the PI for these discussions.
\item
  Researchers are expected to contribute to the lab as a whole, both intellectually and occasionally with requested research from other projects in the lab (potentially outside of a Graduate Student's dissertation project).
\item
  Researchers are expected to follow all of the lab rules that will be given to you when you begin and potentially updated each semester. This includes completing all required ethics training.
\end{enumerate}

\hypertarget{pi-expectations}{%
\section[PI Expectations ]{\texorpdfstring{PI Expectations \footnote{For a full enumeration of PI Advisor Expectations, refer to the document entitled CLL\_GraduateAdvisorExpectations in the lab Dropbox}}{PI Expectations }}\label{pi-expectations}}

\begin{enumerate}
\def\labelenumi{\arabic{enumi}.}
\item
  The PI has the responsibility of meeting regularly with the Graduate Students she advises to discuss their research and academic training. One should give the PI advance notice when scheduling a meeting to find a suitable time.
\item
  The PI has the responsibility of advising students on how to proceed according to their prospective career paths, and will offer continual assistance with research projects that is fully described in the Dropbox document.
\end{enumerate}

\hypertarget{behavior}{%
\section{Behavior}\label{behavior}}

\textbf{Refrain from demeaning, discriminatory, or harassing behaviour and speech.} Harassment includes, but is not limited to:

\begin{itemize}
\tightlist
\item
  deliberate intimidation;
\item
  inappropriate physical contact;
\item
  use of sexual or discriminatory imagery, comments, or jokes
\item
  unwelcome sexual attention.
\end{itemize}

If you feel that someone has harassed you or otherwise treated you inappropriately, please alert the PI or any other member on the team. If any member or project participant engages in harassing behaviour, the senior staff may take any action we deem appropriate.

\hypertarget{communication}{%
\chapter{Communication}\label{communication}}

\hypertarget{slack}{%
\section{Slack}\label{slack}}

Communication within the Child Language Lab should be conducted through Slack, a communication app that uses channels (\#nameofchannel) to communicate to relevant sub-groups or to individuals. Lab members should be familiar with the Slack channels. Using Slack allows us to have work messaging that's separable from personal messaging, and faster and more chat-based than email. This is important for open communicating and the on-the-go aspects of the lab like testing.

\hypertarget{evernote}{%
\section{Evernote}\label{evernote}}

Log all work you do on lab projects in the Evernote notebook associated with that project. Make sure to include the date, your name, and a clear log of what tasks you have completed and notes for any issues that arise. We keep careful diaries of our projects for two main reasons: (a) our methods must be fully replicable down to the smallest decisions, and (b) we will have multiple people working on the same project, sometimes with significant time in between, so work must be able to be picked up both later and by someone else. Make sure Evernote syncs so all work is accessible to everyone in the CLL.

\hypertarget{lab-resources}{%
\chapter{Lab Resources}\label{lab-resources}}

\hypertarget{passwords-shared-credentials-and-room-code}{%
\section{Passwords, Shared Credentials and Room Code}\label{passwords-shared-credentials-and-room-code}}

Long list of passwords and keycodes to all lab accounts are listed in the Dropbox under the `General Lab Documents'. Please keep confidential!

\textbf{If you get locked out:}

\begin{itemize}
\tightlist
\item
  If locked out of 208, Slack someone in the lab for the keycode
\item
  If locked of 209 (Testing room), request Teresa to open the door during work hours. If after work, request the security guard downstairs to open the lab
\item
  PRO TIP: Always make sure the 209 key isn't in the room before locking the door
\end{itemize}

\hypertarget{evernote-1}{%
\section{Evernote}\label{evernote-1}}

Evernote is no longer in use - any important lab documents can be found in the Dropbox or in the Google Drive.

\hypertarget{dropbox}{%
\section{Dropbox}\label{dropbox}}

All materials relevant to lab projects are stored on Dropbox (Usually under Experimental Studies). Each project folder might contain (but limited to) images, audio files, participant tracking spreadsheets, R scripts, data files, and videos related to the project. This is done to ensure a central repository for each project, prevent data loss, allow for cross platform access and remote access. In the future, we plan to create an OSF page and make all our items, data and analysis available to the public to promote open science.

\hypertarget{software-packages}{%
\section{Software Packages}\label{software-packages}}

R, Matlab, Penn Controller, MTurk, Experiment Builder, Data Viewer

\hypertarget{inventory}{%
\section{Inventory}\label{inventory}}

\begin{itemize}
\tightlist
\item
  Lab room cabinet

  \begin{itemize}
  \tightlist
  \item
    1st/bottom drawer: misc
  \item
    2nd drawer: misc
  \item
    3rd drawer: office supplies and toys/gifts
  \item
    4th/top drawer: Kids and adult t-shirts
  \item
    1st shelf/bottom shelf: experiment supplies
  \item
    2nd shelf: kids books
  \item
    3rd shelf: kids toys/gifts
  \item
    4th/top of cabinet: projector
  \end{itemize}
\item
  Cubbies

  \begin{itemize}
  \tightlist
  \item
    kids toys and snacks
  \end{itemize}
\end{itemize}

\hypertarget{onboarding}{%
\chapter{Onboarding}\label{onboarding}}

\hypertarget{basics}{%
\section*{Basics}\label{basics}}
\addcontentsline{toc}{section}{Basics}

\begin{enumerate}
\def\labelenumi{\arabic{enumi}.}
\tightlist
\item
  New hire paperwork: Wasserman Form from Wasserman Center (to Ailis to sign)
\item
  IRB \textgreater{} CITI Training and Certificates \textgreater{} Instructions, for getting IRB
\item
  Create free Slack account; download a copy onto your PC and your phone
\item
  Familiarize yourself with CLL Slack and Dropbox
\item
  Ethics Approval - Complete CITI Training; Certificate saved to Evernote: IRB \textgreater{} CITI Training and Certificates; Certificate saved to DropBox
\item
  Create Cayuse Account
\item
  Get added to active IRB Protocols (Ailis)
\item
  Add email (work/uni) and phone number to General Lab \textgreater{} Contact List
\item
  Learn Passwords (keep confidential!) for:
\item
  Lab Machines: Dropbox; Slack; Google Calendar; Apple/iCloud (includes lab email); Zotero
\item
  Door code (Rm 208) and location of key (Rm 209)
\item
  Scheduling Availability/Hours: Send times you are absolutely not available (class/other work), times you would prefer not to work, and times you are available (prefer to work) to Lab Manager (Charlotte).\\
\item
  Read and sign Rules of the Lab (Dropbox \textgreater{} General Lab Documents)
\end{enumerate}

\hypertarget{experiments}{%
\chapter{Experiments}\label{experiments}}

\hypertarget{how-to-handle-routine-testing-session}{%
\section{How to Handle Routine Testing Session}\label{how-to-handle-routine-testing-session}}

\begin{itemize}
\tightlist
\item
  Alert the security guard before the parents come in. This might involve leaving the name of the parent and age/number of children
\item
  Have consent form ready, explain studies and operation, and sign
\item
  Check Participant Tracker (All Studies) in google drive to see what participant number you are up to and what list you should be running. Note that the participant completed the study on the form afterwards.
\item
  Note on consent form what number participant they are for each study they do. Adult forms all go in the ``completed adult consent forms'' folder. Child forms all go in the completed child forms folder for each respective study; if a child participates in more than one study, make a photocopy of the form, write that they participated in more than one study at the top, and place it one each applicable folder.
\item
  Offer food/snack, be welcoming and accommodating etc
\item
  Run study(ies)
\item
  If adults do \textgreater=2 studies, reimburse afterwards with \$10 (usually found in top drawer in office room, check w/Ailis) and fill out receipt
\item
  If the participant is a child, offer a toy and CLL t-shirt
\item
  As soon as a study is done, add all demographic information to the tracking sheet \& CLL-Database sheet for the respective study. Make sure to mark the date, location (lab or name of daycare), P-number, and child's name.
\end{itemize}

\hypertarget{expov}{%
\section{Experiment Overviews}\label{expov}}

\hypertarget{exhaustivity-alicia}{%
\subsection*{Exhaustivity (Alicia)}\label{exhaustivity-alicia}}
\addcontentsline{toc}{subsection}{Exhaustivity (Alicia)}

Children meet Og, the caveman. Og looks at pictures of animals and objects, and then he describes the pictures, but sometimes he says something that's wrong or a little silly. For example, if Og sees a picture of two blue ducks and one red one, he might say ``There are two blue ducks'' (a correct thing), or he might say ``There are two red ducks'' (a silly thing). The child's job is to reward Og with strawberries to help him learn what a good way to describe the picture versus what a silly way is. When Og does a good job describing the picture, he should get three strawberries. When Og says something silly, he should get one strawberry or maybe two.

\begin{itemize}
\tightlist
\item
  Time: 10-12 minutes
\item
  Age: 4-6
\item
  Materials: Answer sheet, laptop to play video, toy strawberries
\end{itemize}

\hypertarget{count-kiddo-maxime}{%
\subsection*{Count Kiddo (Maxime)}\label{count-kiddo-maxime}}
\addcontentsline{toc}{subsection}{Count Kiddo (Maxime)}

Children are introduced to three `kippies' (fun made-up characters) who want to buy some patterned hats from a wizard. The children are told what each hat costs (hats with stars cost a star, hats with hearts cost a heart and hats with circles cost a circle), and that the kippies will each choose a hat. The kippies then choose a hat and the wizard speaks to one of the kippies. The children are instructed to figure out who the wizard is talking to. They will be told that sometimes the wizard talks about the now, but sometimes he talks about how things could be different instead. The wizard could say something along the lines of ``I wish that kippie's hat had hearts, so he would give me a heart!'' (corresponding image: three kippies, one with a hat with stars, one with a hat with hearts, and one who used to wear a starry hat but took it off for now). A narrator will repeat the utterance and then ask: ``Who's the king talking to?'' After which the child is prompted to tap on the corresponding character displayed on an iPad. In between trials, children will collect stickers/stamps on a separate piece of paper.

\begin{itemize}
\tightlist
\item
  Time: 15-20 minutes
\item
  Age: 3-5
\item
  Materials: Hats and stickers, laptop/tablet to play video, recorder or qualitative answer key
\end{itemize}

\hypertarget{kidbicomp-sarah}{%
\subsection*{KidBiComp (Sarah)}\label{kidbicomp-sarah}}
\addcontentsline{toc}{subsection}{KidBiComp (Sarah)}

Children meet Sylvia, a young Spanish-English bilingual girl who lives next to a farm. Sylvia invites the child to play a game similar to `I Spy', where she looks out her window and says phrases like, ``the dog runs'', ``el perro corre'', or even ``el perro runs.'' The child will see two different images and has to find the image that matches what Sylvia says. The animals and actions vary, but Sylvia describes them the same way.

\begin{itemize}
\tightlist
\item
  Time: 15 minutes
\item
  Age: 2-3
\item
  Materials: eye tracker, questionnaire
\item
  Other: Children must be bilingual Spanish-English
\end{itemize}

\hypertarget{modforcejr}{%
\subsection*{ModForceJr}\label{modforcejr}}
\addcontentsline{toc}{subsection}{ModForceJr}

Children are introduced to Cat (a cartoon cat) and Logan (a puppet who watches the video with the child). Cat is planning a birthday party, and the kids are told that Logan will make statements about the images of Cat running errands, but that Logan doesn't always pay attention, so sometimes he says something silly. The narrator then says something along the lines of ``Cat is going to the bakery to buy a birthday cake for the party! There are two ways to get to the bakery; the red road, and the green road. But uh oh! The green road is blocked'' (corresponding image of a blocked green road and open red road). Then Logan will say ``to get to the bakery, Cat can't go down the green road.'' The narrator asks whether the child thinks Logan is right. That model is repeated a few times, sometimes Logan is right and sometimes he is wrong, and then we see an image of the finished party with some fun music!

\begin{itemize}
\tightlist
\item
  Time: 10 minutes
\item
  Age: 3 ½-4 ½
\item
  Materials: Answer key, iPad
\end{itemize}

\hypertarget{fuzzy-nomnoms}{%
\subsection*{Fuzzy NomNoms}\label{fuzzy-nomnoms}}
\addcontentsline{toc}{subsection}{Fuzzy NomNoms}

Children are shown two animals on the screen that look very similar (e.g., a sheep and a llama), and are then showed an animal in the middle of the screen that is hiding behind the curtain. It looks like it could be either animal and at this point, the child either hears sentences like ``It's maybe a sheep'', or ``It's probably a sheep'', while we track the child's eye-movements to see what they pay attention to on screen in response to such sentences. In this way, we infer the stages along the development path to learning abstract language.

\begin{itemize}
\tightlist
\item
  Time: 30 minutes
\item
  Age: 2-3
\item
  Materials: Eye tracker, headband/clips, cutouts of animals for familiarization
\end{itemize}

\hypertarget{all-about-me}{%
\subsection*{All About ME}\label{all-about-me}}
\addcontentsline{toc}{subsection}{All About ME}

Children are introduced to Pterry the Pterodactyl and Cera the Triceratops, who both comment on pictures using words like must, might and is. The child is then asked ``who said it better,'' and the child comments on whether Pterry or Cera's comment was better.

\begin{itemize}
\tightlist
\item
  Time: 15 minutes
\item
  Age: 5-9
\item
  Materials: answer key, iPad, puppets
\end{itemize}

\hypertarget{cfwaterpark}{%
\subsection*{CFWaterpark}\label{cfwaterpark}}
\addcontentsline{toc}{subsection}{CFWaterpark}

For this task, children will play a short guessing game involving three animals playing in a waterpark (crocodile, penguin, and giraffe). Children are presented with short stories, followed by some questions about those stories. They are asked to guess where giraffe is, based on where the other two animals are, and are asked to explain their choices. This game allows us to learn more about how children reason about possibilities, and understand what kinds of possibilities they are able to reason over.

\begin{itemize}
\tightlist
\item
  Time: 15 minutes
\item
  Age: 5-7 \& adults
\item
  Materials:
\end{itemize}

\hypertarget{squeakymice}{%
\subsection*{SqueakyMice}\label{squeakymice}}
\addcontentsline{toc}{subsection}{SqueakyMice}

Children are invited to play a computer game over zoom with the researcher. The game involves little stories with different animals (mice, bees, ants, fish) playing in different locations. Some of the animals make noises and some do not, and your child's task is to pick which groups of animals to click on to see if they're noisy, based on what the mama animal says. We also ask children to explain their choices, to better understand how they are interpreting the scenario and language. This fun little game helps us learn more about how children understand the kinds of sentences used when we talk about rules and desires/goals.

\begin{itemize}
\tightlist
\item
  Time: 20 minutes
\item
  Age: 5-7 \& adults
\item
  Materials: Scratch, Zoom, computer/laptop, answer key
\end{itemize}

\hypertarget{logistics}{%
\chapter{Logistics}\label{logistics}}

\hypertarget{lab-meetings}{%
\section[Lab Meetings ]{\texorpdfstring{Lab Meetings \footnote{Before the semester begins, the lab manager will contact the lab managers/PI of either lab spaces to check availability and block of a time slot that works best for lab members.}}{Lab Meetings }}\label{lab-meetings}}

We hold lab meetings once a week during the semester and at the discretion of the PI during the summer. Lab meetings usually involve data blitzes, presentations (lab internal and external) and feedback on projects.

\hypertarget{location}{%
\section{Location}\label{location}}

We use either the \textbf{Sociolinguistics Lab} (Floor 3) or \textbf{Phonetics and Experimental Phonology} (PEP; Floor 5).

\hypertarget{contacts}{%
\section{Contacts}\label{contacts}}

\begin{itemize}
\tightlist
\item
  Socio Lab Manager: Kimberly Baxter (\href{mailto:keb565@nyu.edu}{\nolinkurl{keb565@nyu.edu}})
\item
  PEP Lab PI: Dr.~Lisa Davison (\href{mailto:lisa.davidson@nyu.edu}{\nolinkurl{lisa.davidson@nyu.edu}})
\end{itemize}

\hypertarget{snacks}{%
\section{Snacks}\label{snacks}}

The lab manager is entrusted to get snacks for each lab meeting. The lab usually allots a budget of \$20/wk for bagels and coffee. After buying said snacks, the lab manager is reimbursed from the PI's research funds.

\hypertarget{recurring-events}{%
\section{Recurring Events}\label{recurring-events}}

\begin{itemize}
\tightlist
\item
  At the end of each week, RAs and Interns must make sure consent forms are refilled, receipts are catalogued and data is digitized
\item
  They must also print new flyers and put them up around NYU and the environs.
\end{itemize}

\hypertarget{recruiting-and-engagement}{%
\chapter{Recruiting and Engagement}\label{recruiting-and-engagement}}

\hypertarget{recruiting}{%
\section{Recruiting}\label{recruiting}}

Recruiting involves a combination of calling/emailing daycares, parents, and scheduling participants that sign up through the lab's Google form, or contacting individuals who have participated in our research before.

\hypertarget{calling}{%
\subsection*{Calling}\label{calling}}
\addcontentsline{toc}{subsection}{Calling}

Calls should be made from the phone in the lab (Phone \# 212 998 7916). To call, dial 91 followed by the phone number. We primarily call daycares in the 5 boroughs as well as NJ, Long Island and Westchester County.

\begin{itemize}
\tightlist
\item
  Prior to making a phone call, study the calling script (Can be found on the Dropbox under 'All Recruiting \& Testing Docs --\textgreater{} !Use these!'and get a feel for the kind of responses we give.
\item
  Don't memorize the script verbatim at the risk of sounding too robotic. You want to sound friendly and stress that the responsibilities of the day care will be minimal.
\item
  Try not to use the word ``experiment'', as this can sound invasive and can turn people off. Use phrases like ``we are running a study'' or ``we play fun games with kids, and their answers provide us with important information about how they process language''
\item
  It is helpful to shadow a current RA to get a sense of the mechanics of the phone call.
\item
  Once you feel comfortable making a call, you can find a non-exhaustive list of daycares in the CLL-Database
\item
  As you work your way through a list, update contact information if needed and make comments on responses from the contact at the daycare. Comments include:

  \begin{itemize}
  \tightlist
  \item
    If interested, send an email with the required attachments (Institutional Consent, Letter of Cooperation and Consent Forms) which can be found as a template in the lab email. Once you send an email, flag that you have emailed them and make a note to follow up in a week
  \item
    If completely uninterested and do not wish to be called back, flag the day care and make sure not to call again.
  \end{itemize}
\item
  If a daycare responds with the required documents filled out, send the Letter Cooperation to the PI for approval from the IRB.
\item
  Once approved, give a week or two to have the daycare get parents to sign consent forms and then schedule a day to go in.
\end{itemize}

\hypertarget{facebook-outreach}{%
\subsection*{Facebook Outreach}\label{facebook-outreach}}
\addcontentsline{toc}{subsection}{Facebook Outreach}

We should be posting on Facebook regularly (somewhere between 1 and 3 times a week). Try not to post multiple times a day, but rather, spread it out over the week. You can do this by scheduling posts.

Content generally consists of cool/informative videos about kids and language learning, RA spotlights, and announcements about new studies and activities. Look at our previous posts for inspiration, or feel free to start a new series/trend. In general, don't make it too wordy and include visuals if possible! If we have something especially important to post (e.g.~our lab video when it is complete), you can talk to Ailis about paying to ``boost'' the post, so it reaches more users.

Aside from posting on our own page, posting in existing groups makes a BIG difference. Local neighborhood groups, NY parent groups, kids activities groups, and groups for local institutions (local parks, childrens' museums, etc.) are all good places to post. Sometimes this involves applying to the group or contacting the admins of the group directly, which can all be done using the lab facebook page.

\hypertarget{scheduling}{%
\subsection*{Scheduling}\label{scheduling}}
\addcontentsline{toc}{subsection}{Scheduling}

\begin{itemize}
\item
  At the beginning of each semester, RAs should pick out times they can be in the lab and block out those times on the google calendar.
\item
  Once we have the times filled out, the lab manager/RA can fill up those times in the Calendly, the lab's scheduling system.
\item
  When adults or parents of sign up through the google form through the website, they receive an email from the lab account, providing them with a consent form - each RA is responsible for sending these (template emails for each study can be found in the lab gmail account).
\item
  An RA then follows up with the calendly link to schedule a time to come in.
\item
  Once they sign up, both parties will receive a notification and the time slot will be automatically added to the lab calendar.
\item
  A day before testing, the participant will receive a reminder.
\item
  The RA scheduled to work that time slot will be responsible for running the experiment.
\end{itemize}

\hypertarget{offboarding}{%
\chapter{Offboarding}\label{offboarding}}

Give up ID and offload all information linked to your NYU ID to the lab.

\hypertarget{publications}{%
\chapter[Publications ]{\texorpdfstring{Publications \footnote{Adapted from Noor Lab (Noor, 2012)}}{Publications }}\label{publications}}

\begin{itemize}
\item
  Authorship on scientific publications is based on significant contributions to the intellectual input, execution of studies, and writing of papers. Here is the statement on authorship from the journal Evolution: ``Authorship of a paper carries with it responsibility as well as credit. All those whose names appear as authors should have played a significant role in designing or carrying out the research, writing the manuscript, or providing extensive guidance to the execution of the project. They should be able to present and defend the work in a public forum.''
\item
  Within the CLL, the PI (Ailís Cournane) has final say on all authorship issues for work conducted by lab members in whole or in part during one's tenure in the lab. However, as a general guideline, the ``default'' policies are stated below. These are subject to change without notice and to modification for specific circumstances, including but not limited to exceptionally high or minimal contribution in one or more areas.
\item
  For all authorship issues, provisional authorship may be indicated prior to submission, but authorship and author order will not be finalized until the time of manuscript submission, and that decision falls upon the laboratory PI. \textbf{Provisional authorship or author position is no guarantee of authorship or author position on the final submission.}
\item
  Regarding standard research publications, the criterion for co-authorship is contribution of (100/X)\% in any one or combination of the following three areas: intellectual input (including initial design and subsequent modification and development of the project); execution (including data analysis); and writing of the paper (including figure and table preparation) for submission. X denotes the number of potential authors contributing to the study. For example, if one of three potential authors contributes 15\% to all three areas, they may appear as an author of the paper, since their total contribution is 45\% (15\% ´ 3), which is greater than the 33\% (100/3) minimum. We also abide by the Evolution criterion that all authors should be able to present and defend all the work, not just their piece of the work, in a public forum. Such a presentation should include accurate and thorough discussion of the relation of the work to other published studies on the topic. There must have been at least some detectable intellectual input and engagement; \textbf{one cannot have been merely a ``lab-hand.''}
\item
  To be considered for ``first author'', the researcher must have contributed \textgreater50\% of the writing of the final draft as well as (100/X)\% to either the intellectual input or execution of the project. If the paper drafts given to the PI are so poorly prepared (as determined by the PI) that rewriting from scratch is easier than editing, first authorship by the writer is forfeited. The first author has \textbf{no more than 4 months} from the end of data collection for a particular publication, as determined by the lab PI, to prepare the paper draft. Failure to produce a well-prepared draft in this timeframe may forfeit the right to first-authorship.
\item
  For review papers, authorship requires the incorporation of a significant amount of writing, amounting to greater than two paragraphs of text. As above, if a potential author submits text that is either rewritten or otherwise omitted from the final version, authorship may be nullified.
\end{itemize}

\hypertarget{reading-list}{%
\chapter[Reading List ]{\texorpdfstring{Reading List \footnote{Papers without links are in \href{https://drive.google.com/drive/u/0/folders/1e0APGtvOwW9aTcG3hoEQwf31LJStRsO8}{this} dropbox folder}}{Reading List }}\label{reading-list}}

Essential reading list to get up to speed with the general flavor of our research questions.

\hypertarget{modals}{%
\section{Modals}\label{modals}}

\begin{itemize}
\item
  Hacquard, Valentine (2011)
  \href{http://ling.umd.edu//~hacquard/papers/HoS_Modality_Hacquard.pdf}{`Modality'}. In C. Maienborn, K. von Heusinger, and P. Portner (eds.) Semantics: An International Handbook of Natural Language Meaning. HSK 33.2 Berlin: Mouton de Gruyter. 1484-1515.
\item
  Hacquard, V. and Ailís Cournane. 2016. \href{https://s18798.pcdn.co/cournane/wp-content/uploads/sites/5271/2016/09/Hacquard_Cournane_NELS.pdf}{Themes and variations in the expression of modality}. Proceedings of the 46th Annual Meeting of the Northeastern Linguistics Society (NELS 46).
\item
  Barbiers L.C.J. \& Dooren A. van (2017), Modal Auxiliaries. In: Barbiers L.C.J., Dooren A. van (Eds.) The Companion to Syntax.: John Wiley.
\end{itemize}

\hypertarget{acquisition}{%
\section{Acquisition}\label{acquisition}}

\begin{itemize}
\item
  Lidz, Jeff \& L. Perkins (2017) \href{http://ling.umd.edu/assets/publications/Lidz-Perkins-17-LanguageAcquisition.pdf}{Language Acquisition}. in J. Wixted (ed) Stevens Handbook of Experimental Psychology and Cognitive Neuroscience. Wiley.
\item
  Gleitman, L.R., Cassidy, K., Papafragou, A., Nappa, R., \& Trueswell, J.T. (2005) Hard words. Journal of Language Learning and Development, 1:1., 23-64
\item
  Cournane, Ailís. (under revision). Learning Modals: on how children acquire the language of possibility. Language \& Linguistics Compass.
\end{itemize}

\hypertarget{experimental-methods}{%
\section{Experimental Methods}\label{experimental-methods}}

\hypertarget{corpus}{%
\subsection*{Corpus}\label{corpus}}
\addcontentsline{toc}{subsection}{Corpus}

\begin{itemize}
\item
  Cournane, Ailís. 2015. \href{https://s18798.pcdn.co/cournane/wp-content/uploads/sites/5271/2016/09/BUCLD-39-Proceedings_Cournane.pdf}{Revisiting the Epistemic Gap: evidence for a grammatical source}. Proceedings of the 39th annual to the Boston University Conference on Language Development (BUCLD39). Somerville, MA: Cascadilla Press.
\item
  van Dooren, A., Anouk Dieuleveut, Ailís Cournane and Valentine Hacquard. 2017. \href{https://s18798.pcdn.co/cournane/wp-content/uploads/sites/5271/2017/12/AC2017-Proceedings.pdf}{Learning what must and can must and can mean}. Proceedings of the 2017 Amsterdam Colloquium.
\end{itemize}

\hypertarget{sentence-repair-task}{%
\subsection*{Sentence Repair Task}\label{sentence-repair-task}}
\addcontentsline{toc}{subsection}{Sentence Repair Task}

\begin{itemize}
\tightlist
\item
  Cournane, Ailís. 2014. \href{https://www.tandfonline.com/doi/full/10.1080/10489223.2013.855218}{In search of L1 evidence for diachronic reanalysis: mapping modals}. Language Acquisition 21 (1): 103-117.
\end{itemize}

\hypertarget{truth-value-judgment-task}{%
\subsection*{Truth-Value Judgment Task}\label{truth-value-judgment-task}}
\addcontentsline{toc}{subsection}{Truth-Value Judgment Task}

\begin{itemize}
\tightlist
\item
  Gordon, Peter. The Truth-Value Judgment Task. Chapter 10, In D. McDaniel, C. McKee, H. Cairns (Eds.) Methods for assessing children's syntax. Cambridge, Mass: MIT Press.
\end{itemize}

\hypertarget{picture-choice-task}{%
\subsection*{Picture-Choice Task}\label{picture-choice-task}}
\addcontentsline{toc}{subsection}{Picture-Choice Task}

\begin{itemize}
\tightlist
\item
  Gerken, LouAnn \& Michele E. Shady. The Picture Selection Task. Chapter 6, In D. McDaniel, C. McKee, H. Cairns (Eds.) Methods for assessing children's syntax. Cambridge, Mass: MIT Press.
\end{itemize}

\hypertarget{eye-tracking}{%
\subsection*{Eye-Tracking}\label{eye-tracking}}
\addcontentsline{toc}{subsection}{Eye-Tracking}

\begin{itemize}
\tightlist
\item
  Trueswell, J. C., Sekerina, I., Hill, N. M., \& Logrip, M. L. (1999). \href{https://cpb-us-w2.wpmucdn.com/web.sas.upenn.edu/dist/4/81/files/2017/07/1999_Cognition73_89-134-1sw5y3i.pdf}{The kindergartenpath effect: studying on-line sentence processing in young children}. Cognition, 73, 89-134.
\end{itemize}

\hypertarget{language-change-and-the-role-of-the-children}{%
\section{Language Change and the Role of the Children}\label{language-change-and-the-role-of-the-children}}

\begin{itemize}
\tightlist
\item
  Cournane, Ailís. 2017. \href{https://s18798.pcdn.co/cournane/wp-content/uploads/sites/5271/2018/04/In_defence_of_the_child_innovator.pdf}{In defense of the child innovator}. In Mathieu, Éric and Robert Truswell (eds), Micro Change and Macro Change in Diachronic Syntax. Oxford: Oxford University Press. Pp. 10-24.
\end{itemize}

\hypertarget{updating-the-wiki}{%
\chapter{Updating the Wiki}\label{updating-the-wiki}}

\hypertarget{updating-from-docs}{%
\section{Updating from Docs}\label{updating-from-docs}}

If you need to change or information in the wiki, edit the google doc entitled ``Wiki Information''. If you do not update the github page when you update the docs, leave a comment on any changes you made with the date that you made them so that someone can update the website with the changes.

\hypertarget{updating-website}{%
\section{Updating Website}\label{updating-website}}

\begin{enumerate}
\def\labelenumi{\arabic{enumi}.}
\tightlist
\item
  Open the R project ``NYU Child Language Lab Manual.Rproj'' from the folder in downloads called ``CLL-wiki'' in RStudio.
\item
  Locate the .Rmd file in the bottom right panel of RStudio that you wish to edit. If you wish to create a .Rmd file, name it "(chapter number)\_Title" and adjust the subsequent chapter numbers if you created a chapter in the middle of the current chapters. The only formatting requirement for creating a new chapter is to start the .Rmd file with ``\# Title''.
\item
  Edit the text in the .Rmd file. If this requires any formatting refer to the \href{https://www.rstudio.com/wp-content/uploads/2016/03/rmarkdown-cheatsheet-2.0.pdf}{RMarkdown cheatsheet}.
\item
  Save all changes in RStudio. Press the ``Build Book'' button in the top right panel of RStudio, and when the book is previewed, check that your changes look good.
\item
  Open up the terminal\footnote{You may be prompted by the terminal somewhere in this process for our github username and password. These can be found on the passwords page in Evernote.} and type into the command line:
\end{enumerate}

\begin{verbatim}
cd ~/Downloads/CLL-wiki
\end{verbatim}

\begin{enumerate}
\def\labelenumi{\arabic{enumi}.}
\setcounter{enumi}{5}
\tightlist
\item
  You are now in the project directory, which is connected to the github repository for the website. Next, in the order given, enter into the commmand line:
\end{enumerate}

\begin{verbatim}
git add .
\end{verbatim}

The updated files are now in the repository

\begin{verbatim}
git commit -m "Updated (Write Date of Update)"
\end{verbatim}

The version you are uploading is now saved in github.

\begin{verbatim}
git push -u origin master
\end{verbatim}

\begin{enumerate}
\def\labelenumi{\arabic{enumi}.}
\setcounter{enumi}{6}
\tightlist
\item
  The files are now sent to the github repository, and if you check the webpage it should be updated!
\end{enumerate}

\end{document}
